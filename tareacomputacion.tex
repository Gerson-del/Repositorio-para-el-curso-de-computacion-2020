\documentclass[a4paper,10pt]{article}
\usepackage[utf8]{inputenc}

%opening
\title{rumours,album historico.} % Cambia el titulo por alguno que tu elijas.
\author{Gerson Neftaly Juarez Lopez } % Cambialo por tu nombre completo.

\begin{document}

\maketitle

% \begin{abstract}
% 
% \end{abstract}

\section{Seccion 1}

Era el año 1977 y la Fleetwood Mac en su undécimo álbum sorprendia a propios y extraños. El 'Rumours' se convirtio en el disco más vendido de la historia de la música gracias a canciones como "The Chain", "Go your own way", "Dreams", "Dont stop"...

Anoche Rodrigo Contreras reservó una de las mejores habitaciones del Motel para hablar de esta joya en 'El disco del día'.

Fue grabado con la intención de hacer "un álbum pop ", la música del álbum presentaba un sonido pop rock y soft rock caracterizado por ritmos acentuados y teclados eléctricos como el órgano Fender Rhodes o Hammond B3 . Los miembros festejaron y usaron cocaína durante gran parte de las sesiones de grabación, y su finalización se retrasó debido a su proceso de mezcla, pero se terminó a fines de 1976. Tras el lanzamiento del álbum, Fleetwood Mac realizó giras promocionales en todo el mundo. 

'Rumours' obtuvo el reconocimiento generalizado de la crítica, con elogios centrados en su calidad de producción y armonías , que con frecuencia se basaron en la interacción entre tres vocalistas y han inspirado el trabajo de actos musicales en diferentes géneros. Ganó el álbum del año en los vigésimos premios Grammy . Ha vendido más de 40 millones de copias en todo el mundo, lo que lo convierte en uno de los álbumes más vendidos de todos los tiempos . Ha recibido certificaciones Diamond en varios países, incluidos los EE. UU., El Reino Unido, Canadá y Australia.

A menudo considerado el mejor lanzamiento de Fleetwood Mac, el álbum ha aparecido en varias listas de publicaciones de los mejores álbumes de la década de 1970 y de todos los tiempos. \\ % Cambia este texto por algo tuyo.

Los principales escritores de Fleetwood Mac, Buckingham, Christine McVie y Nicks, trabajaron individualmente en canciones, pero a veces compartieron letras entre ellos. "The Chain" es la única canción en la que colaboraron todos los miembros, incluidos Fleetwood y John McVie. Todas las canciones de 'Rumours' se refieren a relaciones personales, a menudo problemáticas. Según Christine McVie, el hecho de que los letristas se centraban en las distintas separaciones se hizo evidente para la banda solo en retrospectiva. "You Make Loving Fun" trata sobre su novio, el director de iluminación de Fleetwood Mac, con quien salió después de separarse de John. "Dreams" de Nicks detalla una ruptura y tiene un mensaje de esperanza, mientras que el esfuerzo similar de Buckingham en "Go Your Own Way" es más pesimista. Después de una breve aventura con una mujer de Nueva Inglaterra, se inspiró para escribir "Never Going Back Again", una canción sobre la ilusión de pensar que la tristeza nunca volverá a ocurrir una vez que se contente con la vida. Las líneas "Estuve abajo una vez / Estuve abajo dos veces" se refieren a los esfuerzos del letrista al persuadir a la mujer de darle una oportunidad. 

"Don't Stop", escrita por Christine McVie, es una canción sobre optimismo. Ella notó que Buckingham la ayudó a elaborar los versos porque sus sensibilidades personales se superponían. La siguiente canción de McVie, "Songbird", presenta letras más introspectivas sobre "nadie y todos" en forma de "una pequeña oración". "Oh Daddy", la última canción de McVie en el álbum, fue escrita sobre Fleetwood y su esposa Jenny Boyd, que acababan de volver a estar juntas. El apodo de la banda para Fleetwood era "The Big Daddy". McVie comentó que la escritura es ligeramente sarcástica y se enfoca en la dirección del baterista para Fleetwood Mac, que siempre resultó ser correcta. Nicks proporcionó las líneas finales "Y no puedo alejarme de ti, bebé / si lo intenté". Su propia canción "Gold Dust Woman" está inspirada en Los Ángeles y las dificultades que se encuentran en esa ciudad. Después de luchar con el estilo de vida del rock, Nicks se volvió adicta a la cocaína; las letras abordan su creencia en "seguir adelante".  \LaTeX.\\ % Cambia este texto por algun texto tuyo.
\section{la identidad de euler}
$ e^{i \pi} + 1 = 0 $  % Comenta esta ecacución y escribe abajo una ecuación muy simple (la que quieras).
Esta ecuacion es la identidad de euler y es bastante importante porque relaciona numeros que son tracendentes.
$\pi $,e,i y el 0
\section{Escribe una ecuacion muy simple}
$F = \frac{Gm1m2}{r^2}$ (Ley de gravitacion universal)
\end{document}


